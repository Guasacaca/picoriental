\chapter{Implementation\label{cha:chapter5}}



\section{Environment\label{sec:env}}
The following software, respectively operating systems, were used for the implementation:

\begin{itemize}
		\item and Ubuntu 6
		\vspace{-0.1in} 
		\item Scala X
		\vspace{-0.1in} 
		\item IntelliJ X
		\vspace{-0.1in} 
		\item Spark X
		\vspace{-0.1in} 
		\item jUnit X
		\vspace{-0.1in} 
		\item ....check POM for more versions
\end{itemize}

\section{General Project Structure\label{sec:projectstructure}}

The Implementation has 4 marked phases that can be seen easily in the pipeline that a test needs to follow.. PICTURE

\section{Preprocessing}
linkedTarget, create virtual categories for the one that contains products: decision of which ID to put them. 
Better structures, titles with AT and DE sometimes...

\section{Feature Modeling}
Which features, what was done
tokens (by empty spaces)
gr\"osse gr\"oße example

attributeTextSearch

 8005 lenght. 
 1. Binary dense vector
 2. Sparse vector of hashes
 3. Concatenate names.

\section{Word2Vector}
Explain the model.

\section{Classification: kNN}
In general KNN

\subsection{Implementation of KNN\label{sec:gui}}

Structure of the algorithm. Screenshot of spark's visualization tool. 
How many shuffles. Etc.

Selection of k. Dinamically (as proposed somewhere), doesnt make any change, because top2 is almost always below suggested threshold and for finding a threshold by our own, would be so time consuming (?)

\lstset{caption=JSON String Code Snippet,label=jsonstring,showstringspaces=false}
\begin{lstlisting}
{
	id: 1,
	method: "myInstance.getGroup",
	params: ["Teammates", 2, true]
}

{
	id: 2,
	result: [
		  "groupDesc":"These are my teammates",
		  {
			"javaClass":"src.package.MemberClass",
			"memberName": "Bob",      
		  }
		]
}\end{lstlisting}

\section{Classification: SVM}
Explain SVM

\subsection{SVM in Spark}
Explain their implementation

\section{One Against All}
Explain what is that, why is it prefered

\subsection{Implementation of OAA}
Explain my implementation

\section{Cross Validation}
Explain my/their implementation

\section{Evaluation}
Evaluation methods explanation

\section{Documentation\label{sec:docu}}

Everything was documented with scaladocs.


